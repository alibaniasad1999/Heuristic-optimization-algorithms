\chapter{مقدمه}
بهینه‌سازی یک فرآیند تصمیم‌گیری است که بیشترین سود از منابع موجود قابل دسترس به‌دست آید. 
مثال‌های ساده از بهینه‌سازی شامل تصمیم‌گیری‌های روزمره، مانند نوع حمل‌ونقل، لباس پوشیدن و خرید مواد غذایی است. برای این وظایف، تصمیم‌گیری می‌تواند بسیار ساده باشد. به عنوان مثال، اکثر مردم‌ارزان ترین حمل و نقل را انتخاب می‌کنند. اکنون شرایطی را در نظر بگیرید که به دلیل برخی شرایط پیش‌بینی نشده، زمانی تا شروع جلسه باقی نمانده است. از آنجایی که سفر با اولین وسیله، با هدف به حداقل رساندن هزینه در تضاد است، انتخاب حمل و نقل بهینه دیگر مانند گذشته ساده نیست و راه حل نهایی نشان دهنده سازش بین اهداف مختلف خواهد بود. این نوع مسائل که شامل در نظر گرفتن همزمان اهداف چندگانه است معمولاً به عنوان مسائل چند هدفه (\lr{Multi-Objective}) شناخته می‌شوند.


بسیاری از مشکلات دنیای واقعی به طور طبیعی شامل بهینه‌سازی همزمان چندین هدف در تضاد است. متأسفانه، این مشکلات دارای اهدافی هستند که در مقایسه با کارهای معمولی که در بالا ذکر شد، بسیار پیچیده‌تر هستند و فضای تصمیم‌گیری اهداف معمولا آنقدر بزرگ است که حل آنها بدون تکنیک های بهینه‌سازی پیشرفته و کارآمد دشوار است. این پروژه به بررسی کاربرد یک روش بهینه‌سازی کارآمد، معروف به بهینه‌سازی \lr{REMARK}، در زمینه بهینه‌سازی چندهدفه می‌پردازد.


برای یک مسأله بهینه‌سازی چند هدفه ساده، احتمال اینکه جواب بهینه‌ای\LTRfootnote{Optimal Solution} یافت شود که به طور همزمان، تمامی توابع هدف تعریف شده در مسأله را بهینه‌سازی کند، بسیار کم است. در بسیاری از موارد، توابع هدف تعریف شده در مسأله بهینه‌سازی چند هدفه با یکدیگر در تناقض هستند. در چنین حالتی گفته می‌شود که برای یک مسأله بهینه‌سازی چند هدفه، جواب‌های بهینه پارتو \LTRfootnote{Pareto Optimal Solutions} وجود خواهد داشت. از لحاظ تئوری، ممکن است بی‌نهایت جواب بهینه پارتو برای یک مسأله بهینه‌سازی چند هدفه وجود داشته باشد.

\section{تاریخچه}
مفهوم برابری\LTRfootnote{Equivalency} عدم فرومایگی\LTRfootnote{Non-Inferiority}\ برای اولین بار توسط ویلفردو پارتو\LTRfootnote{Vilfredo Pareto} و فرانسیس وای. اجورث\LTRfootnote{Francis Y. Edgeworth} و در حوزه اقتصاد معرفی شد. از آن زمان تاکنون، مفهوم بهینه‌سازی چند هدفه، جای پای خود را در حوزه طراحی و مهندسی مستحکم کرده است. ترجمه تحقیقات ویلفردو پارتو ب در سال 1971 منجر به پیاده‌سازی روش بهینه‌سازی چند هدفه در حوزه مهندسی و ریاضیات کاربردی شد. در طول سه دهه اخیر، به‌کارگیری روش‌های بهینه‌سازی چند هدفه در بسیاری از حوزه‌های مهندسی و طراحی به رشد ثابت خود ادامه داده است.


	رویکردهای سنتی برای بهینه‌سازی چند هدفه معمولاً مستلزم تبدیل مسئله اصلی به یک مسئله تک هدفه\LTRfootnote{single-opjective} است. چنین رویکردهایی دارای محدودیت‌های متعددی هستند، از جمله تولید تنها یک راه‌حل برای هر اجرای شبیه‌سازی، نیاز مسئله چند هدفه برای ارضای شرایط کوهن تاکر و حساسیت به شکل جبهه پارتو\LTRfootnote{Pareto front}. از سوی دیگر، استفاده از رویکردهای فراابتکاری\LTRfootnote{Metaheuristic} که از پدیده های اجتماعی، بیولوژیکی یا فیزیک الهام گرفته شده اند، مانند الگوریتم مبتنی بر رفتار انسان (\lr{REMARK})، بهینه‌سازی ازدحام ذرات (\lr{PSO})، الگوریتم تکاملی (\lr{EA})، سیستم ایمنی مصنوعی (\lr{AIS})، تکامل دیفرانسیلی (\lr{DE})، و تبرید شبیه‌سازی شده (\lr{SA}) در سال‌های اخیر به عنوان جایگزین‌های انعطاف‌پذیرتر و مؤثرتر برای حل مسائل بهینه‌سازی پیچیده، افزایش یافته‌است.
	
	بهینه‌سازی چند هدفه یک موضوع تحقیقاتی چالش برانگیز است، نه تنها به این دلیل که شامل بهینه‌سازی همزمان چندین هدف پیچیده در مجموعه بهینه پارتو می‌شود، بلکه باید بسیاری از مواردی که منحصر به مسائل چند هدفه هستند، مانند تخصیص تناسب جواب‌ها \cite{article_Farina, Tan2008}
	 ، حفظ تنوع\LTRfootnote{diversity preservation} \cite{Khor2005}، تعادل بین اکتشاف و بهره‌برداری\LTRfootnote{exploration and exploitation} 
	 \cite{10.1007/978-3-540-31880-4_30} و نخبه گرایی\LTRfootnote{elitism} \cite{laumanns2000a} توجه کنند. بسیاری از الگوریتم‌های مختلف
	  \lr{CA, PSO, EA, AIS, DE} و SA
	   برای بهینه‌سازی چند هدفه از تلاش‌های پیشگام شافر \cite{schaffer1985multiobjective} با هدف پیشرفت در زمینه‌های ذکر شده، پیشنهاد شده‌اند. همه این الگوریتم‌ها در روش‌شناسی و همچنین، در تولید راه‌حل‌های جدید، متفاوت هستند.

\section{ الگوریتم‌های مبتنی بر رفتار انسان}
در مقوله‌ی الگوریتم‌های بهینه‌سازی، به تازگی الگوریتم‌های فراابتکاری مبتنی بر انسان توسعه داده شده‌اند که فعالیت‌های اجتماعی و تعاملات انسانی را مدل می کند.
قضیه بهینه‌سازی \lr{no free lunch} بیان می‌کند که هیچ تضمینی وجود ندارد که الگوریتم بهینه‌سازی که بتواند مسئله خاصی را حل کند، در بهینه‌سازی‌های دیگر به خوبی عمل کند. این دلیل اصلی برای توسعه الگوریتم‌های بهینه‌سازی جدید است.

الگوریتم‌های فراابتکاری مبتنی بر رفتار انسان بر اساس مدل‌سازی ریاضی فعالیت‌های مختلف انسانی که فرآیندی مبتنی بر تکامل دارند، معرفی می‌شوند. بهینه‌سازی مبتنی بر یادگیری\LTRfootnote{Teaching-Learning-Based Optimization} (\lr{TLBO}) معروف‌ترین الگوریتم مبتنی بر رفتار انسان است که بر اساس شبیه‌سازی ارتباط و تعامل بین معلم و دانش‌آموز در کلاس طراحی شده است
\cite{RAO2011303}.
 در طراحی بهینه سازی فقیر و غنی\LTRfootnote{Poor and Rich Optimization} (\lr{PRO}) \cite{SAMAREHMOOSAVI2019165},
فعالیت های اقتصادی افراد غنی و فقیر در جامعه ایده اصلی بوده است. در طراحی جستجوی ذهنی انسانی\LTRfootnote{Human Mental Search} (\lr{HMS})  \cite{Mousavirad2017} از شبیه‌سازی رفتار انسان در برابر بازارهای حراج آنلاین برای دستیابی به موفقیت استفاده شده است. در طراحی\LTRfootnote{Doctor and Patient Optimization} \lr{DPO} \cite{app10175791} از تعامل بین پزشکان و بیماران از جمله پیشگیری از بیماری، چک آپ و درمان استفاده شده است.

