
% -------------------------------------------------------
%  Abstract
% -------------------------------------------------------


\pagestyle{empty}

\begin{وسط‌چین}
\مهم{چکیده}
\end{وسط‌چین}


\پرش‌بلند
\بدون‌تورفتگی بسیاری از مسائل دنیای واقعی شامل بهینه‌سازی همزمان چندین هدف با محدودیت‌های مختلف است که حل آنها بدون کمک الگوریتم‌های بهینه‌سازی ابتکاری، اگر غیرممکن نباشد، دشوار است. آنچه بهینه‌سازی چند هدفه را بسیار چالش‌برانگیز می‌کند این است که در صورت وجود اهداف متناقض، راه‌حل بهینه‌ای برای همه‌ی اهداف نیست و الگوریتم‌های بهینه‌سازی باید قادر به یافتن تعدادی راه‌حل باشد که بتوان آنها را جایگزین یکدیگر کرد و بین اهداف مصالحه\LTRfootnote{trade off} کرد.  با این وجود، چند هدفی یکی از جنبه‌های بهینه‌سازی در دنیای واقعی است.
الگوریتم بهینه‌سازی \lr{REMARK} یک روش جستجوی تصادفی است، که در حل مسائل پیچیده کارآمد و مؤثر است.
از مزیت  \lr{REMARK} می‌توان به رویکردهای مبتنی بر ازدحام، عرضه و تقاضا اشاره کرد. این رویکرد باعث ارتباط اعضای جمعیت با یکدیگر می‌شود، که به همگرایی سریعتر و بررسی مکان‌های مستعدتر در فصای جست‌وچو منجر می‌شود. اهمیت دیگر این رویکرد در بهینه‌سازی چندهدفه این است که، هر گروه از جمعیت یک قسمت از مجوعه پارتو\LTRfootnote{Pareto set} را بررسی می‌کند و این اطمینان را می‌دهد که مجموعه پارتو با تقریب بالایی بررسی می‌شود.

 \مهم{کلیدواژه‌ها}: 
 الگوریتم‌های بهینه‌سازی، چندهدفه، جمعیت، داد و ستد، مجموعه پارتو

\صفحه‌جدید
