\section{سوال اول}
برای تبدیل این موضوع ابتدا باید تابع هزینه و قیود را بررسی کرد. یکی از توابع هزینه رایج را می‌توان زمان تأخیر مسافران را در نظر گرفت.
این تابع هزینه بستگی به نوع هواپیما و تعداد مسافر داخل آن دارد. برای مثال هر دقیقه تأخیر یک بوئنگ ۷۴۷ بالاتر از بوئنگ ۷۷۷ است.
از طرفی مسأله دارای قید نیز هست؛ به طور مثال، زمانی که یک اتفاق اضطراری رخ می‌دهد و یا سوخت یکی از هواپیما‌ها تمام می‌شود حتما باید فرود بیاید وگرنه هزینه به شدت بالا می‌رود. قید دیگر را می‌توان حداکثر تأخیر در نظر گرفت، به طور مثال، یک مسافر نباید بیشتر از یک ساعت تأخیر را تحمل کند.
قید دیگر را می توان بسته به نوع هواپیما فرود آمده بررسی کرد که تا چه زمانی و چه هواپیما‌هایی قابلیت نشستن دارند.
 در شرایط مختلف آب‌و‌هوایی ممکن است تابع هزینه و قیود نیز عوض شوند.
\subsection{قسمت اول}
در این حالت چون صرفا یک باند وجود دارد هزینه و قید مانند بالا تعریف می‌شود و در هر لحظه تمامی حالت ها بررسی می‌شود و هواپیما انتخاب می‌شود.