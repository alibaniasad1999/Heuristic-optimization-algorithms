
% -------------------------------------------------------
%  Abstract
% -------------------------------------------------------


\pagestyle{empty}

\begin{وسط‌چین}
\مهم{چکیده}
\end{وسط‌چین}

\بدون‌تورفتگی در این پژوهش، از یک روش مبتنی بر نظریه بازی\LTRfootnote{Game Theory} به‌منظور کنترل وضعیت استند سه درجه آزادی چهارپره استفاده شده‌است. 
%در این روش سیستم و اغتشاش دو بازیکن اصلی در نظر گرفته شده‌است. هر یک از دو بازیکن سعی می‌کنند امتیاز خود را  با کمترین هزینه افزایش دهند که در اینجا،  وضعیت استند امتیاز بازیکن‌ها در نظر گرفته ‌شده‌است. 
در این روش بازیکن اول سعی در ردگیری ورودی مطلوب می‌کند و بازیکن دوم با ایجاد اغتشاش سعی در ایجاد خطا  در ردگیری بازیکن اول می‌کند.
در این روش انتخاب حرکت با استفاده از تعادل نش\LTRfootnote{Nash Equilibrium} که با فرض بدترین حرکت دیگر بازیکن است،  انجام می‌شود.
این روش نسبت به اغتشاش ورودی و همچنین نسبت به عدم قطعیت مدل‌سازی  می‌تواند مقاوم باشد.
% از روش ارائه‌شده برای کنترل یک استند سه درجه آزادی چهارپره که به نوعی یک آونگ معكوس نیز هست، استفاده شده‌است. 
برای ارزیابی عملکرد این روش ابتدا شبیه‌سازی‌هایی در محیط سیمولینک انجام شده‌است و سپس، با پیاده‌سازی روی استند سه درجه آزادی صحت عملکرد کنترل‌کننده تایید شده‌است. 

\پرش‌بلند
\بدون‌تورفتگی \مهم{کلیدواژه‌ها}: 
چهارپره،  بازی دیفرانسیلی، نظریه بازی، تعادل نش، استند سه درجه آزادی، مدل‌مبنا، تنظیم‌کننده مربعی خطی 
\صفحه‌جدید
