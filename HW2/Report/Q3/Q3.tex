\section{سوال سوم}
برای بهینه‌سازی از روش
\lr{SA}
استفاده شده است.
تابع دما به صورت زیر در نظر گرفته شده است.
$$
T_{k+1} = \alpha T_{k}
$$

در اجرای کد دو حالت ۱۰ و ۱۰۰  زنجیره مارکو در اجرا شده است و نتایج در جدول در فصل ؟؟؟؟؟؟؟؟؟؟؟
آورده شده است. برای یافتن همسایگی از عوض کردن خانه‌های رشته جواب استفاده شده است. تعداد خانه‌های عوض شده تابعی از دما است. به این صورت، از حداکثر ۴ خانه تا حداقل ۲ خانه انجام شده است. پارامتر‌های انتخاب شده برای بهینه‌سازی با
\lr{random sampling}
شامل دمای اولیه، دمای نهایی و ضریب $\alpha$ است.
بازه جواب‌ها در جدول \ref{tab:parameters} آمده است.

\lr{\begin{table}[!h]
		\caption{Parameter of SA algorithm}
		\vspace{0.5cm}
		\centering
		\begin{tabular}{|c|c|c|c|}
			\hline
			 $\alpha$ & $T_f$ & $T_0$  &  Parameter
			\\ \hline
			$0.99$ & $10^0$ & $10^5$ & $\max$ \\
			$0.90$ & $10^{-4}$ & $10^2$ & $\min$
 		  \\ \hline
		\end{tabular}
	\label{tab:parameters}
\end{table}}



\begin{table}[!h]
		\caption{نتایج اجرا با مجموعه‌های مختلف}
		\vspace{0.5cm}
		\centering
		\begin{tabular}{|c|c|c|c|}
			\hline
			\lr{Set 3} & \lr{Set 2} & \lr{Set 1}  &  
			\\ \hline
			$20$ & $20$ & $20$ &
			 تعداد اجراهای موفق \\
		$125.5$ & $124.7$ & $122.1$ &
		 میانگین کمترین زمان امدادرسانی\\
				$6791$ & $8497$ & $7805$ &
				 میانگین تعداد ارزیابی تا نخستین دستیابی به بهترین جواب
			\\ \hline
		\end{tabular}
\end{table}

نمودار دما بر حسب تابع هزینه برای بهترین پارامتر‌های بدست آمده رسم شده است.


