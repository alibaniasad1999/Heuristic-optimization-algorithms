\section{سوال دوم}
در این سوال برای بهینه‌سازی از دو روش
\lr{Random Search}
و
\lr{Simplex}
استفاده شده‌ است.
برای بدست‌آوردن مقدار بهینه ضریب انبساط از روش
\lr{Quadratic Interpolation}
استفاده شده است. این روش برای بهینه‌سازی سه عدد احتیاج دارد. در این راه حل با بررسی حالت‌های مختلف ضریب انبساط 0/5 انتخاب شد و برای پیدا کردن دو نقطه دیگر ضرایب مختلفی در آن ضرب شد که نتیجه آن در ادامه آورده شده است. از طرفی، ضریب انبساط ثابت نیست و به مرور مقدار آن کمتر شده تا بهینه‌سازی عملکرد بهتری داشته باشد.
در روش  \lr{Random Search}
از تولید رشته‌ای از اعداد رندوم استفاده شده است، به این دلیل که، سرعت اجرای برنامه بیشتر می‌شود. نتایج اجرای برنامه‌ها در فایل‌های \lr{csv}
آورده شده است.



همانطور که در بخش قبل گفته شد ضریب انبساط به صورت \lr{greedy}،
۰/۵ در نظر گرفته شد. در برنامه وقتی که در یک مرحله تابع هزینه کم نشد ضریب انبساط در ۰/۹ ضرب می‌شود که برنامه بتواند عملکرد بهتری داشته باشد. در 
\lr{Quadratic Interpolation}
برای دو عدد دیگر، ضریب انبساط درصدی افزایش و کاهش می‌یابد که نتیجه آن در جدول 
\ref{tab:alpha}
آورده شده است.



\newpage




